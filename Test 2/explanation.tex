\documentclass[11pt,twoside,a4paper]{article}
\usepackage{ctex}
\usepackage{amsfonts}
\usepackage{listings}

\begin{document}
    \title{explanation}
    \author{olxh}
    \date{22\,.5\,.3}

    \maketitle
    \section{K近邻}
    \subsection{K近邻思路}
        K近邻法(k-nearest neighbor,k-NN)存在一个样本集合,当我们对新的数据进行比较的时候,我们选取前K个进行比较。通常K是不大于20的整数,最后,选出K个最相似数据中出现次数最多的分类,作为我们新数据的分类。
    
        对于每一个样本,我们有一个$n$维度的向量$v$,向量里面的值代表了这个样本中$n$个不同的特征值,当我们进行比较的时候,我们直接通过比较两个向量之间的差距即可。
        
        此外,对于每一个样本,它都有一个自己所属的分类。

        我们可以运用类似两点之间的距离公式计算距离。我们定义两个样本之间的“差距”为:

        对于两个样本A和B而言,它们之间的“差”可以用下面这个公式来表示
        \begin{displaymath}
            |AB| = \sqrt{ \sum_{i=1}^{n} (A_i-B_i)^2 }
        \end{displaymath}
        对于一个只有两个特征的样本,这个公式我们应该很熟悉,就类似两点之间的距离公式。

        接着,对于一个新加入的样本,我们可以分别求出它与其它各个样本的差距,由小到大选出前K个进行计算。通过找出这K个样本里面出现最多的分类来确定当下这个点的所属分类。


    \section{代码的实现}
    \subsection{准备数据集}
        下面是一段准备数据集的代码:
        \begin{lstlisting}[language={Python}]
            def createDataSet():
            group = np.array([[1,101],[5,89],[108,5],[115,8]])
            labels = ['1','1','2','2']
            return group,labels
        \end{lstlisting}

    \subsection{主函数处理}
        接着是处理的主函数
        \begin{lstlisting}[language={Python}]
            def classify0(inX,dataSet,labels,k):
            #shape[0] means the rows
            dataSetSize = dataSet.shape[0]
            #repeat the array
            diffMat = np.tile(inX, (dataSetSize, 1)) - dataSet

            #count the differences between two vectors
            sqDiffMat = diffMat ** 2
            sqDistances = sqDiffMat.sum(axis = 1)
            distances = sqDistances ** 0.5
            sortedDistIndices = distances.argsort()


            #count the apperances of each label
            classCount = {}
            for i in range (k) :
                voteIlabel = labels[sortedDistIndices[i]]
                classCount[voteIlabel] = classCount.get(voteIlabel,0) + 1
            
            #itemgetter was used for sorting
            #reverse means sorting from big to small ones
            sortedClassCount = sorted(classCount.items(),key=operator.itemgetter(1),reverse = True)

            return sortedClassCount[0][0]
        \end{lstlisting}
    

        \section{不足与改进}
            因为训练集数量过少,现在我们得到的结果并不是很令人满意,我们可以通过提高数据的训练强度来达到我们的目的。

\end{document}